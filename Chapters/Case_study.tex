\chapter{Case study} \label{sec:Case study}
The main purpose of this section is to inform the reader about the state of affairs at the time of this project. The feasibility of a drainage flow forecasting model had not been considered before and so it was unknown weather the data available would be sufficient for the task. Many decision were made for which time series to use and which to leave out since the introduced some new or mostly unrelated problem into the project. However these shortcomings in the data may be present for other wastewater utilities and so this is to document how these were handled or why they were left out. 

\subsection{Drainage flow data}
\textbf{Problem 1: Hourly averages}
Throughout this project the drainage data flow data is treated as though it were equal to the input into the drainage system or pumping station when in reality it is actually the output from the pumping stations. This small assumption means that we ignore a possible discrepancy between the two caused by the difference of the water level between the beginning and the end of the hour which can easily be more than 30\% higher or lower than the input during the same hour. The critical difference between the two is that the output is only equal to the input plus or minus the change in volume stored in the chamber within the station within the pumping station. The height of the water level within the station is continuously monitored and then the average water level over the duration of an hour is saved as well as the maximum water level and the minimum water level. 

It goes without saying that same average height per hour can be produced by very dissimilar circumstances. For example if for 30 minutes the water chamber was full and then for 30 minutes it was empty, the average would be half the maximum height. The same height can be achieved with exactly half the maximum height persisting throughout the hour. But for these two circumstances, if we assume the flow into the pumping stations was the same, the difference between the average flow out of the station is equal to the volume of the chamber divided by one hour. 

In order to realize the effect of this, the following numbers are true for the Boðagrandi pumping station. The area of the cross-section of the chamber is about $50m^2$ so each metre of height difference means about $51'000 l$ of volume in the chamber. During 2020 the hourly mean flow-rate was about $60l/s$ with the minimum observed hourly mean was $25l/s$ and the maximum value was $141 l/s$. During dry weather conditions the system has 1-2 of the two pumps running at a time and the system is designed in such as way that the pumps are always at either full power or off. The second pump is turned on when the water level reaches $1.8m$ and off when it reaches 1.15m. This normal cycle can repeat during the course of an hour. Let's assume the chamber is empty at the start of the hour and the true flow into the chamber is about $60l/s$ and so during the hour $216'000 l$ pass into the chamber. If the chamber is at the $1.15m$ mark at the start of this period and at the $1.8m$ at the end of the hour then the mean volume that has flowed out of the chamber during that hour is equal to $60 l/s * 3600 - (1.8-1.15) * 50'000 = 183'500 l$. If we assume the opposite case where the chamber is at $1.8m$ at the start and $1.15m$ at the end then the flow out of the chamber during the hour is equal to $60 l/s * 3600 + (1.8-1.15) * 50'000 = 248'500 l$, about $29\%$ higher than the other case. If we assume that the minimum value during the past year, 25l/s which is about 90'000 l during the hour, is a product of this exact scenario then the input flow during that hour could have been about $90'000 + (1.8-1.15)*50'000$ or about $122'500$, which is roughly $36\%$ higher than the observed output flow. Of course this kind of error averages out over a long time and will be approximately centered at the true value but such error can still effect the evaluation of the error for such functions. This error was just one example of the kinds of errors that can stem from this kind of data problem even in ideal conditions and since this chambers operating range is between $0.5m$ and $3.35m$ the difference in flow during an hour could theoretically exceed $142'500l$ which translates to about $39.6l/s$ difference between the input flow and output flow, which is about $34\%$ of the difference between the maximum flow and minimum flow during the year 2020. The key information here is that without information about the water level at the start and end of each hour in addition to the hourly output flow or measurements of the input flow, it's impossible to be sure how much water flowed into the station during that hour. 

\textbf{Problem 2: Missing data during overflow}
One of the primary objectives with most drainage flow forecasting attempts is to reduce or prevent overflow in the wastewater treatment plants, pumping stations or the sewers themselves. So when it comes to the importance of data probably the most valuable data is when the system is under close-to or beyond maximal load. However in the case of Reykjavíks drainage system almost all pumping stations  will have some sort of backup system in place for releasing untreated sewage and wastewater out of the pumping stations. In these cases it's unknown how much water exits through the backup channel. During this time the flow meters at the normal exit will usually stop or will be flagged in the SCADA system as having low quality or being invalid or will show nothing at all. This has the result that real-world performance of the model can't be assessed beyond a certain threshold of flow. 

\subsection{Radar data}
The radar data used in this project contains radar echo data over a 6 year period starting at the begining of 2015 to the end of 2020 collected and supplied by the Icelandic Meteorological Office. The radar in question is the ISKEF radar in the Southern Peninsula Region. The data used in this project is from the start of 2015 to the end of 2020. During this time several radar strategies were used, of which two were the most common and can be seen in figures \ref{fig:radarstrat1} and \ref{fig:radarstrat3}. 

(HERE ID LIKE TO HAVE A DIAGRAM THAT SHOWS AN EXAMPLE OF THIS WITH PERHAPS A 2X8 GRID SHOWING THE RAW RADAR DATA AND THEN A SINGLE CAPPI IAMGE BELOW)


% The raw radar echo images generated by Doppler weather radar are noisy due to factors like ground clutter, sea clutter, anomalous propagation and electromagnetic interference [18]. To alleviate the impact of noise in training and evaluation, we filter the noisy pixels in the dataset and generate the noise masks by a two-stage process described in the appendix.
The same two-stage process described in \cite{shi2015convolutional} is used to reduce noise during training and evaluation. This noise is assumed to come from factors such as ground clutter, sea clutter, anomalous propagation and electromagnetic interference.

The first part of this processing involves removal of static interference such as ground clutter and sun spikes by means of an unsupervised outlier removal technique described below. 

For each pixel, count the number of instances where a pixel is given a particular value. This forms a 255 dimensional vector for each  pixel. Then compute the mean [insert formula here] and covariance matrix [insert formula here] across all pixels. Then the Mahalanobis distance [insert formula here] from the estimated mean is computed for each pixel. Pixels with a distance more than 3 standard deviations of the distance away from the mean are considered outliers. 

After this all values smaller than 71 and larger than 0 are removed to reduce the effect of factors such as sea clutter.

These processing steps are primarily for preparing the data for the training of the precipitation nowcasting algorithm. 

The nowcasting algorithm is primarily 






\subsection{Rain-gauge data}

\subsection{Temperature forecast data}
In order to forecast using a model that depends on the temperature it is necessary to use temperature forecasts instead of temperature observations in order to get a realistic view of the performance that would be seen when actually forecasting. The forecast data used in this project originates from the same NWP model as the precipitation forecast data but is specifically estimated for at the location of the 1475 weather station where the temperature observations are also made. 





